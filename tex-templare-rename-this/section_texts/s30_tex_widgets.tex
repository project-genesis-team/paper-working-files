
\section{section - example} 
		
\subsection{subsection - example}
Here's everything you could ever want to know!

%figure
\begin{figure}[h] % [t], [b], [p]
\centering
\includegraphics[width=4cm]{./images/cat_sip.jpg}
\caption{wait...what?}
\end{figure}

\section{section - das ist das...}
...text2 file

\subsection{itemize (subsection)}

	\begin{itemize}
		\item this is item 1
		\item and this is item 2
	\end{itemize}
	
\subsection{theorems (subsection)}
		
	\label{mytheorem}
		Let $G$ and $H$ be Lie groups with Lie algebras ...
		Then:
		\begin{enumerate}
			\item blabla nr. 1
			\item blabla nr. 2
		\end{enumerate}

\underline{Proof}:
here we prove theorem with the number \ref{mytheorem}.

\section{section}

\subsection{hyperlink, book reference, footnote (subsubsection)}

\url{http://en.wikipedia.org/wiki/LaTeX}

this is a ref with the number \cite{testBook}.

And now follows a footnote. 
\footnote{here is the footnote}

\begin{center}
  \begin{tabular}{{|} c {|} c {|} c {|} c {|}} % four centered text blocks
    \hline
    1 & x & 7 & hi \\ \hline % the four textblock contain the data 1, x, 7 and hi
    4 & 5 & 6 & 6 \\ \hline
    7 & 8 & g & 6 \\
    \hline
  \end{tabular}
\end{center}

\begin{center}
  \begin{tabular}{ l {|} c {|}{|} r r {|}} % now also using left, right and double seperators
    \hline
    1 & 2 & 3 & 3 \\ \hline
    4 & 5 & 6 & 6 \\ \hline
    7 & 8 & 9 & 6 \\
    \hline
  \end{tabular} 
\end{center} 

% variant 1
	\begin{equation}
	\label{testEquation}
		a^2+b^2=c^2
	\end{equation}

Verweis auf Formel mit der Nummer (\ref{testEquation}).

% variant 2
	\[
		g_{\mu\mu'} g_{\nu\nu'} \varepsilon^{\mu'\nu'\rho\sigma} 
		= \det(g_{\mu\nu}) \varepsilon_{\mu\nu\rho'\sigma'} g^{\rho\rho'} g^{\sigma\sigma'}
	\]
	
% variant 3
	$$
		g_{\mu\mu'} g_{\nu\nu'} \varepsilon^{\mu'\nu'\rho\sigma} 
		= \det(g_{\mu\nu}) \varepsilon_{\mu\nu\rho'\sigma'} g^{\rho\rho'} g^{\sigma\sigma'}
		= G
	$$

		